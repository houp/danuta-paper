\documentclass[a4paper,pdflatex,sn-mathphys-num]{sn-jnl}% Springer Nature article style

\usepackage{microtype}%
\usepackage{amsmath,amssymb,amsfonts}%
\usepackage{amsthm}%
\usepackage{mathrsfs}%
\usepackage{xcolor}%
\usepackage{orcidlink}%

\theoremstyle{thmstyleone}%
\newtheorem{theorem}{Theorem}%
\newtheorem{proposition}[theorem]{Proposition}%

\theoremstyle{thmstyletwo}%
\newtheorem{example}{Example}%
\newtheorem{remark}{Remark}%

\theoremstyle{thmstylethree}%
\newtheorem{definition}{Definition}%

\begin{document} 

\title{A Decade of the Gdańsk School of Cellular Automata}

\author[1]{\fnm{Barbara} \sur{Wolnik}\,\orcidlink{0000-0003-2935-5529}}
\author*[2]{\fnm{Witold} \sur{Bołt}\,\orcidlink{0000-0001-6787-2100}}\email{Witold.Bolt@jit.team}
\author[1]{\fnm{Antoni} \sur{Augustynowicz}\,\orcidlink{0000-0002-0748-4867}}
\author[2]{\fnm{Aleksander} \sur{Bołt}\,\orcidlink{0000-0002-4888-2722}}
\author[3]{\fnm{Marcin} \sur{Dembowski}\,\orcidlink{0000-0002-5824-9779}}
\author[1]{\fnm{Maciej} \sur{Dziemiańczuk}\,\orcidlink{0000-0002-0553-7750}}
\author[4]{\fnm{Adam} \sur{Dzedzej}\,\orcidlink{0000-0002-1974-7927}}
\author[1]{\fnm{Dominik} \sur{Falkiewicz}\,\orcidlink{0000-0001-7755-7745}}
\author[1]{\fnm{Bartosz} \sur{Makuracki}\,\orcidlink{0000-0003-1102-6321}}
\author[1]{\fnm{Anna} \sur{Nenca}\,\orcidlink{0000-0003-2746-1061}}
\author[5]{\fnm{Aleksander} \sur{Wardyn}}

\affil[1]{\orgdiv{Faculty of Mathematics, Physics and Informatics}, \orgname{University of Gdańsk}, \orgaddress{\street{Wita Stwosza 57}, \city{Gdańsk}, \postcode{80-308}, \country{Poland}}}
\affil[2]{\orgname{Jit Team}, \orgaddress{\street{Łużycka 8C}, \city{Gdynia}, \postcode{81-537}, \country{Poland}}}
\affil[3]{\orgname{Xebia Poland}}
\affil[4]{\orgname{Gemini}}
\affil[5]{\orgname{CodeComply.Ai}}

\abstract{The Gdańsk School of Cellular Automata (GSCA) has, over the last decade, developed a coherent program that bridges theory with exploratory computation to study various aspects of Cellular Automata. In this paper, we review the key results of the group on (i) identifying cellular automata from partial, noisy observations, (ii) affine continuous cellular automata and their links to stochastic and deterministic rules, (iii) density classification, (iv) parity classification, and (v) number-conserving cellular automata across uniform, non-uniform, and non-square grids, including reversibility and split-and-perturb decompositions. This survey is written to honor the memory of prof.\,Danuta Makowiec who directly motivated the creation of GSCA and influenced the research methodology used by the group.}

\keywords{cellular automata, affine continuous cellular automata, system identification, density classification problem, parity problem, number-conserving cellular automata, reversibility, non-uniform cellular automata}

\maketitle

\section{Introduction}\label{sec:introduction}
The scientific contributions of prof.\, Danuta Makowiec to the theory and applications of Cellular Automata (CAs) are widely recognized through her numerous research projects, publications, and conference contributions.
Beyond her outstanding academic achievements, prof.\, Makowiec was also an exceptional teacher and mentor, deeply respected by her students and collaborators alike.
She delivered lectures and conducted computer laboratories within courses on ``Cellular Automata'' and ``Complex Systems'', inspiring generations of students through her intellectual passion and personal dedication.

A distinguishing feature of her academic activity was a teaching philosophy rooted in the harmonious integration of rigorous mathematical reasoning with computer simulation and \emph{experimentation}.
While such a methodology is well established in the physical sciences, it represented an innovative and formative experience for young mathematicians and computer scientists attending her courses.
For many of them, this approach not only shaped their understanding of complex systems, but also decisively influenced their scientific identity.

It was this unique educational atmosphere—and, above all, the research paradigm uniting theory with experimentation—that motivated one of prof.\, Makowiec's former computer science students, Witold Bołt, and a theoretical mathematician, Barbara Wolnik, to pursue systematic investigations into the properties of CAs.
This partnership gave rise to the Gdańsk School of Cellular Automata (GSCA).
The research carried out within GSCA focuses primarily on theoretical aspects of CAs, with particular emphasis on number-conserving rules, global computational capabilities (notably density and parity classification), and the problem of rule identification.
Over the past decade, members of GSCA have published more than 30 peer-reviewed articles and presented their results at more than 20 international conferences and workshops.
Moreover, as a direct outcome of research conducted within the group, two members defended their master's theses (Aleksander Bołt and Aleksander Wardyn), three members completed their PhD dissertations (Adam Dzedzej, Witold Bołt, and Marcin Dembowski), and finally, Barbara Wolnik obtained her habilitation.

This article surveys and synthesizes the results obtained by GSCA.
Although prof.\, Makowiec was not directly involved in the work reported here, her profound and lasting influence on the group—as a mentor, colleague, and source of inspiration—permeates the research presented in this paper.
In this sense, the work discussed in this article stands as a natural continuation of her scientific legacy.

The primary aim of this article is to honor the memory of prof.\, Danuta Makowiec and to express our deepest gratitude for her inspiration, mentorship, and enduring influence.
We dedicate this work to her with respect, admiration, and heartfelt remembrance.

\section{Identification of CAs}\label{sec:identification}
We built an identification pipeline that recovers CA rules from partial or noisy space--time traces. Evolutionary search handles bounded time gaps in incomplete observations \cite{bot2015CEC201,bolt2020TCYB20}, while frequency-based estimators address $\alpha$-asynchronous and diploid rules and can impute missing cells \cite{bot2016978331,bot2017978331,bot2019jbiosy}.

\section{Affine Continuous CAs}\label{sec:accas}
Affine continuous CAs (ACCAs) can be expressed as convex combinations of classical deterministic rules, which links them conceptually to stochastic mixtures and clarifies how continuous dynamics relate to stochastic CA behavior \cite{bot2015jjocs2}. Legal outer-totalistic ACCAs exhibit rich dynamics and multiple modes of sensitivity to parameters, grid size, and single-site perturbations \cite{wolnik2022s11071}. These properties underpin later constructions for density and parity classification tasks.

\section{Density Classification Problem}\label{sec:dcp}
Our work began with the relaxed $\alpha$-DCT formulation for continuous CAs \cite{bolt2014}. Finite-size ACCAs solve fixed-length instances \cite{dembowski2017s11047}, and most density-conserving ACCAs of radius one succeed on relaxed variants \cite{wolnik2017aa7d86,wolnik2016978331,8617353}. Together these results frame ACCAs as a viable alternative to classical binary rules for density classification.

\section{Parity Problem}\label{sec:parity}
We closed long-standing gaps on the minimal neighborhood for parity classification: no one-dimensional CA with a six-cell neighborhood solves parity \cite{NENCA2024114923}. A proof-backed fix of the BFO construction reinstates a single-rule solution, confirming that parity is solvable by one CA rule \cite{wolnik2025cellularautomatareallysolve}.

\section{Number-Conserving CAs}\label{sec:ncca}
This line spans characterization, decomposition, reversibility, and non-uniform settings. Survey chapters summarize the landscape and open problems \cite{wolnik2025978303,wolnik2025978303b}. Geometric conditions characterize von Neumann rules and show binary number-conserving CAs are intrinsically one-dimensional \cite{wolnik2017aa89cf,wolnik2019PhysRe}. Split-and-perturb methods yield complete lists and reversible families, including four-state 2D and higher-state cases \cite{dzedzej2019ab25df,wolnik2020jphysd,dzedzej2020jic202,wolnik2020jins20,dzedzej2021jins20,wolnik2022jphysd}. Rotation-symmetric exploration uncovers a seven-state example beyond septenary isomorphism and establishes dimensional thresholds \cite{wolnik2023PhysRe,8685839}. Extensions to triangular grids provide decompositions and enumerations \cite{wolnik2023jtcs20}, while non-uniform elementary rules are fully characterized on finite and infinite grids \cite{wolnik2023jins20,wolnik2023jins20b}. Reversibility on triangular and hexagonal grids is settled via constructive counterexamples and complete tests \cite{wolnik2021500315,wolnik2021s02181,8579003}.



\bibliography{papers-with-abstracts}

\end{document}
