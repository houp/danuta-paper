%!TEX TS-program = xelatex
\documentclass[pdflatex,sn-mathphys-num]{sn-jnl}% Springer Nature article style

\usepackage{microtype}%
\usepackage{amsmath,amssymb,amsfonts}%
\usepackage{amsthm}%
\usepackage{mathrsfs}%
\usepackage{xcolor}%
\usepackage{subfig}
%\usepackage{orcidlink}%

\newcommand{\ie}{\emph{i.e.}, } 

\begin{document} 

\title{Danuta Makowiec and the rise of Gdansk School of Cellular Automata}

\author[1]{\fnm{Barbara} \sur{Wolnik}}%\,\orcidlink{0000-0003-2935-5529}}
\author*[2]{\fnm{Witold} \sur{Bołt}}\email{Witold.Bolt@jit.team}%,\orcidlink{0000-0001-6787-2100}
\author[1]{\fnm{Antoni} \sur{Augustynowicz}}%\,\orcidlink{0000-0002-0748-4867}
\author[2]{\fnm{Aleksander} \sur{Bołt}}%\,\orcidlink{0000-0002-4888-2722}
\author[]{\fnm{Marcin} \sur{Dembowski}}%\,\orcidlink{0000-0002-5824-9779} %% no affil !!!
\author[1]{\fnm{Maciej} \sur{Dziemiańczuk}}%\,\orcidlink{0000-0002-0553-7750}
\author[3]{\fnm{Adam} \sur{Dzedzej}}%\,\orcidlink{0000-0002-1974-7927}
\author[1]{\fnm{Dominik} \sur{Falkiewicz}}%\,\orcidlink{0000-0001-7755-7745}
\author[1]{\fnm{Bartosz} \sur{Makuracki}}%\,\orcidlink{0000-0003-1102-6321}
\author[1]{\fnm{Anna} \sur{Nenca}}%\,\orcidlink{0000-0003-2746-1061}
\author[1]{\fnm{Adam} \sur{Rutkowski}}%\,\orcidlink{0000-0002-8393-8170}

\affil[1]{\orgdiv{Faculty of Mathematics, Physics and Informatics}, \orgname{University of Gdansk}, \orgaddress{\street{Wita Stwosza 57}, \city{Gdansk}, \postcode{80-308}, \country{Poland}}}
\affil[2]{\orgname{Jit Team}, \orgaddress{\street{Łużycka 8C}, \city{Gdynia}, \postcode{81-537}, \country{Poland}}}
\affil[3]{\orgname{Gemini Polska}}

\abstract{The Gdansk School of Cellular Automata (GSCA) has, over the last decade, developed a coherent program that bridges theory with exploratory computation. We review results on (i) identifying cellular automata from partial, noisy observations, (ii) affine continuous cellular automata (ACCAs) and their links to stochastic and deterministic rules, (iii) relaxed density classification using ACCAs, (iv) parity classification bounds and constructive single-rule solutions, and (v) number-conserving and state-conserving cellular automata across uniform, non-uniform, and non-square grids, including split-and-perturb decompositions, graph-based enumerations, reversible subclasses, and traffic-like particle-flow rules. The survey is written in the spirit of prof.\,Danuta Makowiec’s emphasis on combining rigorous analysis with computational experimentation, and aims to serve both as a guide for newcomers and as a blueprint for future collaborative work.}

\keywords{cellular automata, affine continuous cellular automata, system identification, density classification problem, parity problem, number-conserving cellular automata, state-conserving cellular automata, non-uniform cellular automata, reversibility, traffic flow}

\maketitle

\section{Introduction}\label{sec:introduction}
The scientific contributions of prof.\,Danuta Makowiec to the theory and applications of Cellular Automata (CAs) are widely recognized through her numerous research projects, publications, and conference contributions.
Beyond her outstanding academic achievements, prof.\, Makowiec was also an exceptional teacher and mentor, deeply respected by her students and collaborators alike.
She delivered lectures and conducted computer laboratories within courses on ``Cellular Automata'' and ``Complex Systems'', inspiring generations of students through her intellectual passion and personal dedication.

A distinguishing feature of her academic activity was a teaching philosophy rooted in the harmonious integration of rigorous mathematical reasoning with computer simulation and \emph{experimentation}.
While such a methodology is well established in the physical sciences, it represented an innovative and formative experience for young mathematicians and computer scientists attending her courses.
For many of them, this approach not only shaped their understanding of complex systems, but also decisively influenced their scientific identity.

It was this unique educational atmosphere and, above all, the research paradigm uniting theory with experimentation, that motivated one of prof.\,Makowiec's former computer science and mathematics students, Witold Bołt, to continue working on CAs after finishing his studies. Thanks to the collaboration with the KERMIT Research Unit from Ghent University, led by prof.\ Bernard De Baets, the first research project of the yet-to-be-formed group started. This in turn quickly led to building broader interest in studies on CAs in Gdansk, with Barbara Wolnik -- a theoretical mathematician from Gdansk -- being the key person building up the team and taking the role of a research lead. This partnership gave rise to the Gdansk School of Cellular Automata (GSCA).

The research carried out within GSCA focuses primarily on theoretical aspects of CAs, with particular emphasis on number-conserving rules, global computational capabilities (notably density and parity classification), and the problem of rule identification. Over the past decade, members of GSCA have published more than 30 peer-reviewed articles and presented their results at more than 20 international conferences and workshops.
Moreover, as a direct outcome of research conducted within the group, two members defended their master's theses (Aleksander Bołt and Aleksander Wardyn), three members defended their PhD dissertations (Adam Dzedzej, Witold Bołt, and Marcin Dembowski), and finally, Barbara Wolnik obtained her habilitation.

This article surveys and synthesizes the results obtained by GSCA so far. Although prof.\,Makowiec was not directly involved in the work reported here, none of it would happen without her. Her profound and lasting influence on the group as a mentor, colleague, and source of inspiration, permeates the research presented in this paper. In this sense, the results presented in this article stand as a natural continuation of her scientific legacy. By presenting this article, we want to honor the memory of prof.\,Danuta Makowiec and express our deepest gratitude for her inspiration, mentorship, and enduring influence. We dedicate this work to her with respect, admiration, and heartfelt remembrance.

In the following sections, the key results of GSCA are reported. Throughout the manuscript we assume the reader has a general understanding of the theory of CAs, therefore the basic definitions are omitted. Interested readers can find a good introduction to CAs and related topics in \cite{sep-cellular-automata} or in a much more detailed form in the recently published taxonomy of CAs \cite{ROLLIER2025108362}.

\section{Identification of CAs}\label{sec:identification}

The first research project, which became the building block of GSCA, was started by Witold Bołt under the supervision of Jan Baetens and Bernard De Baets from Ghent University. Its topic is the identification problem for CAs. In the identification problem, based on a set of observations, potentially incomplete (missing cells or time steps) and noisy (some states wrongly observed), of a behavior of an unknown CA, we aim at identifying the rule of this CA. See Figure \ref{fig:observation} for an example of three types of observations: complete (no missing cells), for which identification is trivial and thus not considered; spatially partial, where only some cells are missing; and partial, where not only individual cells but also entire time steps are missing. We studied and solved two special cases of this identification problem. First, we considered binary, deterministic, one-dimensional CAs. Second, we focused on non-deterministic CAs. For these two sub-problems different solution strategies have been proposed.

\begin{figure}[ht]
\centering
\subfloat[]{\includegraphics[width=0.3\columnwidth]{rule30-50x50-np.pdf}}\hspace{0.5cm}
\subfloat[]{\includegraphics[width=0.3\columnwidth]{rule30-50x50-sp.pdf}}\hspace{0.5cm}
\subfloat[]{\includegraphics[width=0.3\columnwidth]{rule30-50x50-tsp.pdf}} 
\caption{Example of a (a) complete, (b) spatially-partial and (c) partial observations of ECA 30 (time goes from top to bottom, grey cells correspond to ``unknown state'').}
\label{fig:observation}
\end{figure}

In the case of deterministic CAs, an evolutionary approach was used to search the solution space efficiently. The evolutionary algorithm designed for the problem is based on the original genetic algorithm (GA) in which the local rules of CAs represented as look-up table (LUT) vectors are evolved. In contrast with classical GA, in our algorithm it is allowed to have individuals of variable length in one population. Thanks to this, the algorithm is able to correctly identify the neighborhood radius of the unknown local rule. In addition, various techniques for speeding up the fitness function computation are given in case of big observation sets, inspired by classical machine learning optimization techniques. The initial problem setting and solution strategy is given in \cite{bot2015CEC201}, while the full detailed algorithm description along with numerous results from computational experiments are reported in \cite{bolt2020TCYB20}. One of the experimental results obtained allowed us to link the identification effort (number of expected evolutionary iterations) with the complexity of the unknown CA measured with Wolfram's complexity classes and maximal Lyapunov exponents. Our algorithm can successfully uncover the CA from partial observations as complex as the one shown on Figure \ref{fig:observation} in point (c).

The case of non-deterministic CAs is significantly different from the deterministic case, and thus required different solution strategies. Due to the stochastic nature, the problem is more challenging, and thus we focus only on spatially partial observations as seen in Figure \ref{fig:observation} point (b). We considered two specific subgroups of Stochastic CAs (SCAs) in this line of research. Namely, $\alpha$-asynchronous CAs ($\alpha$-ACAs) and Diploid CAs were considered. Both of these classes are built from deterministic CAs. To define an $\alpha$-ACA we take a deterministic CA and at each time step, for every cell independently, we make a decision if the CA's rule will be applied (with probability $\alpha$) or not (with probability $1-\alpha$). Diploid CA is a natural extension of this concept -- we take two deterministic CAs, and independently for each cell, we make a selection of one of them to apply.

In \cite{bot2016978331} we present a statistical method, based on frequencies, for identifying $\alpha$-ACAs from partial observations (some cells having an unknown, missing state). The presented method, in addition to finding the underlying deterministic CA and estimating the value of probability $\alpha$ is able to unveil the missing state values with high accuracy. In \cite{bot2017978331} we focus on the identification of diploid CAs. The identification method presented in this paper is an extension of the method presented in~\cite{bot2016978331}. Here we focused only on finding the underlying deterministic CA rules and numerical parameter. The observations were complete (no missing states). Finally, \cite{bot2019jbiosy} is a continuation of the work on the identification of diploid CAs started in \cite{bot2017978331} -- the case of incomplete observations was discussed here. The goal of the identification algorithm is to estimate the parameters of the underlying diploid CA and to estimate the missing states in the observations. 

The presented results form the first step towards establishing a general identification method, based on incomplete observations, for SCAs, as any SCA can be expressed in a form of a stochastic mixture of a finite number of deterministic CAs, whereas in diploid CAs we allow only for mixing two such CAs. (In a separate research track of our team, in \cite{bot2015jjocs2} we give an algorithm for decomposing a SCA to a specific stochastic mixture. Although such a decomposition is not unique, our algorithm allows us to uncover the deterministic component with the application probability which is highest among all possible decompositions.)

\section{Affine Continuous CAs and DCP}\label{sec:accas}

While studying the identification problem of deterministic CAs, described in the previous section, numerous attempts were made to obtain a computationally efficient and stable evolutionary algorithm. In one of the tested approaches, we tried to ``approximate'' the unknown deterministic CA with a SCA, which later on was modified to use non-classical CAs that could be parameterized as a vector of real numbers from the unit interval, just as SCAs. As an outcome of this (failed) experimentation effort, our attention turned to the more general studies of a specific class of real-valued CAs, which nowadays we refer to as Affine Continuous CAs (ACCAs). 
ACCAs are the simplest possible such generalization of binary CAs, as they have a local rule that is affine in each variable, and as a result, they are simply convex combinations of classical binary CAs. Although it might seem that such a generalization does not contribute anything significant to the behavior of the resulting dynamical systems, this is not the case. See Figure~\ref{fig:cca} for an example of a space-time diagram of an ACCA.

\begin{figure}[h]
\centering
\includegraphics[height=0.5\textwidth]{cca.pdf}
\caption{Example of a space-time diagram of an ACCA, where time runs from top to bottom and states are assigned a color following the presented color scale.}
\label{fig:cca}
\end{figure}

ACCAs from one point of view have a relatively simple formulation and can be seen as the most natural generalization of finite-state CAs to the continuous state space. On the other hand, surprisingly, they were rarely studied in the past. To better understand their properties, in \cite{bot2015jjocs2} we built a more general form of ACCAs as a generalization of multi-state CAs, with the use of $[0,1]^n$ state space, and studied some of their connections with SCAs. In \cite{bolt2014} we proposed a novel experiment, which turned out to open multiple new research directions. We tried to experimentally find ACCAs that solve the Density Classification Problem (DCP) with the use of evolutionary algorithms. In DCP we seek a CA that, for an arbitrary (binary) initial configuration of odd length, reaches a final state of all-0s or all-1s, depending on the initial density of 1s (number of 1s in the initial configuration). It is known from the literature that DCP cannot be solved with CAs \cite{PhysRevLett.74.5148}. The early experimental results in \cite{bolt2014} were very promising and led GSCA to continue the efforts in two separate, yet connected tracks. First, further study of the DCP in the context of ACCAs. Second, to study specific, detailed properties of ACCAs relevant to the DCP, with strongest focus on number-conserving CAs (CAs conserving sum of states), which is covered in a separate section of this article, and legal, outer-totalistic CAs (CAs whose rules depend on the central cell state and the sum of states of its neighborhood).

Our novel research showed that legal outer-totalistic ACCAs exhibit rich dynamics and multiple modes of sensitivity to parameters, grid size, and single-site perturbations \cite{wolnik2022s11071}. 
Thanks to massive numerical simulations, we have been able to partition the rule space into classes with distinct behavior. A unique combination of computer simulations (sometimes quite advanced) and a panoply of analytical methods allowed us to lay bare the dynamics of each and every one of these ACCAs and confirm all results theoretically. In particular, the results obtained established that the dynamics of ACCAs is much richer than their classical counterparts. These properties underpin later constructions for density and parity classification tasks.

In \cite{dembowski2017s11047} we showed that ACCAs can successfully solve DCP for fixed-size systems (fixed maximal number of cells) in one dimension, yet a full solution to the DCP in the traditional sense is not possible just like traditional CAs. Yet, we were able to show that with a very simple relaxation of the DCP problem, we can find general solutions with ACCAs. Our idea was very simple: since we use ACCAs, \ie $[0,1]$ is the state set, we don't need to converge to all-0s or all-1s configuration to have an unambiguous answer. We may consider a ``relaxed'' formulation in which the output needs to be ``all cells less (or greater) than $0.5$'' depending on the initial density. Knowing the detailed characterization of one-dimensional density-conserving ACCAs with radius one obtained in~\cite{wolnik2022s11071}, it was relatively easy to show that almost all of them solve this relaxed DCP~\cite{wolnik2016978331,wolnik2017aa7d86}. This result was further extended to the two-dimensional case in \cite{8617353} where ACCAs with von Neumann neighborhood were used.

\section{Parity Problem}\label{sec:parity}

The success of progressing the frontier of state-of-the-art knowledge on the classical DCP, which has been studied in the CA theory for years, motivated us to consider other classical classification problems. Aside DCP, the second best-known of such problems seems to be the parity problem. The classical parity problem asks whether a CA can correctly classify any initial configuration (with an odd number of cells) according to whether it contains an even or odd number of 1s. As with DCP, the difficulty lies in performing this global classification using only local interactions. Research conducted in this area aims to find answers to two key questions. First, does a binary CA that solves the parity problem exist? Second, if it does exist, how complex must it be, where the measure of complexity is the neighborhood size $d$?

The initial results obtained by various researchers were very promising, but at the same time somewhat turbulent. In 2013, a paper appeared proposing the first known solution of the parity problem: the BFO rule with $d=9$~\cite{BFO2013}. Although the underlying idea of the BFO rule was sound, its specification -- both the look-up table (LUT) and the Wolfram code -- turned out to be incorrect.
Proponents of this rule later identified this issue themselves and provided several new candidate rules to solve the parity problem. However, no rigorous mathematical proof has been provided for any of these rules.
Nevertheless, it has been established that no binary CA with neighborhood size at most five (\ie\ $d\leq 5$) can solve the parity problem.
Numerous computational searches have also been conducted to find a CA with $d=7$ that can solve the parity problem~\cite{Wolz2008}, all have been unsuccessful (there have even suspicions that $d=7$ is not enough).

As the first step in this research field, we published a proof that no binary CA with a six-cell neighborhood ($d=6$) can solve the parity problem~\cite{NENCA2024114923}.
(Sadly, the proof is highly technical and cannot be easily generalized to larger neighborhoods, therefore $d>6$ was still an unknown territory.)

As mentioned above, the only well-known solution candidate was the BFO rule, yet its formulation contained a mistake which made some researchers believe that the parity problem, similarly to DCP, cannot be solved by CAs.
After deep further analysis of the BFO rule, we were able to diagnose and correct the flaw in the LUT from the original paper. What is more important, in collaboration with the author of the original BFO, we presented a rigorous mathematical proof demonstrating that the corrected BFO rule actually solves the parity problem~\cite{wolnik2025cellularautomatareallysolve}. Figure~\ref{fig:BFO} shows the action of the corrected BFO on sample configurations.

\begin{figure}[!ht]
\begin{minipage}[b]{0.45\textwidth}
\centering
\includegraphics[width=0.9\textwidth]{nBFO0.pdf}\\
(a)
\end{minipage}
\begin{minipage}[b]{0.45\textwidth}
\centering
\includegraphics[width=0.9\textwidth]{nBFO1.pdf}\\
(b)
\end{minipage}
\caption{Action of the corrected BFO on two sample configurations with a length of 1313, which differ by only one bit (time runs from top to bottom).}
\label{fig:BFO}
\end{figure}

Thus, now it is known that a nine-cell neighborhood is sufficient to design a binary CA that solves the parity problem, while six cells are insufficient. A gap remains, however: for CAs with neighborhood sizes $d=7$ and $d=8$, no definitive results are currently available.

Moreover, in the course of our research, another question emerged: if we were to relax the requirements of the classical parity problem by ``half'' -- that is, ask for a CA that can solve it only for all configurations with an odd number of ones or only for all configurations with an even number of ones -- would the task become significantly easier for CAs? By ``significantly easier'' we mean that there would exist a binary CA with a smaller neighborhood size that solves such a simplified parity problem. In~\cite{nenca2025} we show that no binary CA with a neighborhood of size $d=6$ can solve such a relaxed problem.

Although this still does not answer the question of whether simplified parity problem is really easier for CAs than the classical one, we hope that the final answer is now within reach. Namely, we have found a collection of binary CAs with $d=7$ that seem to be perfect candidates to solve the classical parity problem. Due to our experiments, we already know that these CAs correctly verify all configurations of odd length up to 31 cells. Up to now we have no formal mathematical proof, but we are working on it actively. Our preliminary investigation in this direction leads us to put forward a bold conjecture that there is a CA with $d=7$ that solves the classical parity problem. If we can prove that this conjecture is true, we will conclude all of the above threads. Not only will we know that CAs can solve the parity problem, but we will also know that the simplest solutions require only 7 cells in the neighborhood. What is more, it will turn out that, contrary to expectations, the simplified parity problem is just as difficult (or easy) for CAs as the classical one.

\section{Number-Conserving CAs}\label{sec:ncca}

The most effort of GSCA so far went into researching number-conserving CAs. This topic emerged from our earlier interest in solving the DCP, which is closely related to number conservation.
Number conservation is an important property of CAs and means that the sum of all states in any configuration remains constant throughout the evolution of the system.
Among the results in this topic, those concerning multidimensional CAs with the von Neumann neighborhood definitely prevail.
First, using a novel approach based on a geometric analysis of the von Neumann neighborhood in higher dimensions (see Figure~\ref{fig:vN}), we proposed a family of necessary and sufficient conditions for a $d$-dimensional CA to be number conserving, formulated in terms of the local rule~\cite{wolnik2017aa89cf}. These conditions apply for any state set of real numbers, whether it is finite or not, and can be formulated in 
$(2d+1)\cdot 2^{d^2}$ different but equivalent ways.
Thus in each particular case we can choose the version that suits us best, so this is especially useful to describe number-conserving CAs satisfying some additional conditions -- like, for example, rotation symmetry (see, for instance, \cite{dzedzej2021jins20, 8685839, wolnik2023PhysRe}).

\begin{figure}[!ht]
\centering
\begin{minipage}[b]{0.31\textwidth}
\centering
\includegraphics[width=0.9\textwidth]{nbd1d.png}\\
(a)
\end{minipage}
\begin{minipage}[b]{0.31\textwidth}
\centering
\includegraphics[width=0.9\textwidth]{nbd2d.png}\\
(b)
\end{minipage}
\begin{minipage}[b]{0.31\textwidth}
\centering
\includegraphics[width=0.9\textwidth]{nbd3d.png}\\
(c)
\end{minipage}
\caption{The von Neumann neighborhood in the case of (a) $d=1$ (b) $d=2$ (c) $d=3$.}
\label{fig:vN}
\end{figure}

The next step in our research in this topic was to develop a very useful mathematical tool for studying multidimensional number-conserving CAs with the von Neumann neighborhood: the split-and-perturb decomposition theorem~\cite{wolnik2020jphysd}.
This theorem states that the local rule of any number-conserving CA with the von Neumann neighborhood can be decomposed into two parts:  a~split function and a perturbation.
Moreover, the set of all possible split functions has a~very simple structure, while the set of all perturbations forms a linear space and is therefore very easy to describe in terms of a basis.

Both these approaches turned out to be quite easy to implement on a~computer and we were able to find the complete list of all $d$-dimensional number-conserving CAs with the von Neumann neighborhood and the state set $Q$ in many cases of $d$ and $Q$ that had been previously completely beyond the computational capabilities of computers (see, for instance, \cite{dzedzej2019ab25df, dzedzej2020jic202, wolnik2020jphysd, wolnik2022jphysd, data-dziemianczuk-2020, data-dzedzej-2020, data-nenca-2022}).

Additionally, it turned out that the split-and-perturb decomposition theorem can also be a convenient tool for proving some general properties concerning $d$-dimensional number-conserving CAs with the von Neumann neighborhood (never seen before in the literature). 
For example, it was possible to fully uncover the structure of binary such CAs and it appeared that regardless of the dimension~$d$, all of these CAs are trivial in the sense that there are exactly $4\,d+1$ $d$-dimensional binary number-conserving CAs with the von Neumann neighborhood: the identity rule, the shift rules, and traffic rules in each of the $2\,d$ directions~\cite{wolnik2019PhysRe} (so they all are intrinsically one-dimensional). 
Moreover, the split-and-perturb decomposition theorem allowed us to characterize all reversible ternary $d$-dimensional number-conserving CAs with the von Neumann neighborhood~\cite{wolnik2020jins20}. Unfortunately it appeared
that any such CA is a shift (in some of $2\,d$ directions).

The method introduced in~\cite{wolnik2020jphysd} for number-conserving CAs defined on regular $d$-dimensional cubic grids, was later adapted to study number conservation of two-dimensional CAs defined on a regular triangular grid and updating the states of their cells on the basis of the states of the adjacent cells only~\cite{wolnik2023jtcs20}.
In contrast to the case of cubic grids, in the case of a triangular grid, the split-and-perturb decomposition approach allows to enumerate all $k$-ary such CAs, regardless of the value of $k$. 

A part of our research focuses on an important subclass of number-conserving CAs, the so-called \emph{state-conserving} CAs. Generally speaking, being state-conserving simply means that for each initial configuration the number of cells in each state is constant throughout the entire evolution of the system. Our investigation allowed us to characterize all one-dimensional $k$-ary state-conserving CAs with radius one and we were able to enumerate them all, regardless of the value of $k$ (see Figure~\ref{fig:sc}). Moreover, from the aforementioned characterization, it was also very easy to deduce that among them there are only three reversible ones: the identity and the shift rules~\cite{stateco2025}.

\begin{figure}[!ht]
\centering
\includegraphics[width=0.9\textwidth]{state-conserving.png}
\caption{There is one-to-one correspondence between the set of all state-conserving one-dimensional CAs with state set $Q$ and with radius one that are not shift rules and the set of all labeled directed graphs with $k$ vertices and not containing a directed path of length two. Here we see a space-time diagram of some such CA and the graph that corresponds to it.}
\label{fig:sc}
\end{figure}

Recently, our research has also covered $d$-dimensional number-conserving binary CAs with the Moore neighborhood. We are still at the simulation stage in this area, but we already have a very useful tool (the so-called $(\Omega,\Lambda)$ framework describing movements of ones interpreted as particles) that allows us to design hundreds of thousands of interesting two-dimensional number-conserving binary CAs with radius one that have nontrivial dynamics~\cite{falkiewicz2025}. Some of these CAs, despite fulfilling the very binding condition of being number-conserving, exhibit elements of behavior such as the famous Game of Life (see, for example, CA $D$ in Figure~\ref{fig:D}).

\begin{figure}[!ht]
\centering
\includegraphics[width=0.9\textwidth]{fig_06.pdf}
\caption{An example of an oscillator that occurs in the evolution of one of the two-dimensional number-conserving binary CAs with Moore neighborhood (the so-called CA $D$ described in~\cite{falkiewicz2025}).}
\label{fig:D}
\end{figure}

The $(\Omega,\Lambda)$ framework also makes it possible to follow the motion of individual particles. One notable implication of this is that the period of an oscillator can be redefined: instead of the usual definition, it can be taken as the time it takes for all individual particles to return to their original positions. An example of this is shown in Figure~\ref{fig:Dcolors}, which presents an oscillator in CA $D$ whose period differs under the classical and the new approach.

\begin{figure}[!ht]
\centering
\includegraphics[width=0.9\textwidth]{fig_07.pdf}
\caption{Another example of an oscillator that occurs in the evolution of CA $D$. Since the $(\Omega,\Lambda)$ framework introduced in~\cite{falkiewicz2025} allows us to track the movement of each particle separately, we can mark them with different colors. This example shows that although the arrangement of particles returned to its initial ``shape'', not every particle returned to its original position.}
\label{fig:Dcolors}
\end{figure}

We anticipate that this line of investigation will be brought to completion in the near future. The research described above also raised many new and interesting questions and yielded several unexpected results that seem to contradict earlier intuitions (see, for instance the survey papers~\cite{wolnik2025978303, wolnik2025978303b}. All this shows that we still do not fully understand these seemingly very simple dynamic systems.

\section{Non-uniform CAs}\label{sec:non-u}

All of the research topics described so far related to the classical CA formulation which assumes spatial homogeneity -- each cell is governed by the same rule. Yet, from an application perspective this assumption may be sometimes too limiting, while from the theoretical perspective it is very interesting to check how important this homogeneity is when it comes to key properties of CAs. This motivated us to extend our focus to study so-called non-uniform CAs.
The central focus of this work is the number-conserving property of such dynamical systems. Currently, the research is limited to one-dimensional binary automata, with plans to broaden the scope in the future.
To date, we have successfully identified and thoroughly characterized the dynamics of all number-conserving non-uniform elementary CAs (in which individual cells are allowed to have its own local updating rule belonging to Wolfram's set of $256$ elementary local rules), considering both finite grids and the infinite grid.

In the case of finite grids, it has been shown that, irrespective of whether periodic or null boundary conditions are considered, any such CA is either a classical uniform number-conserving elementary CA or it is a conglomerate of components, which can be of four different types only (see Figure~\ref{fig:n-u}). Moreover, these components exhibit very limited dynamical behavior, since they evolve independently of one another, as if they were separated by impermeable boundaries. Consequently, for all such non-uniform CAs, every configuration eventually reaches a fixed point or becomes periodic with period two~\cite{wolnik2023jins20}.

\begin{figure}[!ht]
\centering
\includegraphics[width=0.8\textwidth]{non-uniform.pdf}
\caption{Illustration of the evolution of some number-conserving non-uniform elementary CA on a finite grid (with periodic boundary conditions). Each such CA is either a classical (uniform) number-conserving elementary CA or it is a conglomerate of components, which can be of four different types only ($Id$, $S$, $A_n$ and $B_n$) and each of them has a very poor dynamics. Moreover, each component of the conglomerate ``lives its own life'', as if there were impermeable barriers between adjacent components.}
\label{fig:n-u}
\end{figure}

This characterization made it possible, among other results, to enumerate all number-conserving non-uniform elementary CAs, as well as those that are reversible. It was shown that for a grid consisting of $n$ cells, the number of number-conserving non-uniform elementary CAs is $F_n^2$, while the number of reversible ones is $F_n$, where $F_n$ denotes the $n$-th Fibonacci number.

The case of the infinite grid came as a big surprise. Previously, it was believed that the answer to what number-conserving non-uniform elementary CAs on the infinite grid look like could be obtained by performing simulations on finite grids of increasing length. However, it turned out that, unlike when considering number conservation for classical CAs, the infinite grid cannot be treated as a limiting case of finite grids, \ie there are number-conserving non-uniform CAs on the infinite grid that have no analogous counterpart on finite grids~\cite{wolnik2023jins20b}.

At present, we are investigating one-dimensional non-uniform binary CAs with four-cell neighborhoods~\cite{makuracki2025, makuracki2026}. As is common in the theory of CAs, increasing the neighborhood size by only one cell leads to doubly exponential growth in complexity, rendering this case substantially more involved and significantly more demanding from a computational perspective. Nevertheless, the investigation has already reached an advanced stage, and a comprehensive characterization of this class of CAs is expected in the near future.

\section{Summary}\label{sec:final}

In this paper we summarized the key results of our broad research on CAs -- starting from the identification problem, going forward with ACCAs, DCP, the parity problem, number and state conservation, and finally non-uniform CAs. In all these research directions, we were able to advance to state-of-the-art by uncovering new properties, introducing new tools, and forming new generalizations or simplifications. Despite the fact that some of the topics may initially seem disconnected, we showed how they naturally emerged from each other, building upon earlier experiments and discoveries. In all of the considered topics, the key element of success was the one that was given to us by prof.\ Danuta Makowiec -- the combination of experimentation with a rigorous analytical approach. This culture of work was possible due to the formation of a truly interdisciplinary team of computer scientists, mathematicians and physicians working together, supported by modern computation techniques including high-performance computing (HPC) solutions. This interplay between theory and computation not only enabled us to obtain the results presented in the preceding sections, but also allowed us to identify and critically reassess several erroneous results previously reported in the literature~\cite{wolnik2021500315,wolnik2021s02181,8579003}. 

\section*{Acknowledgments}
The authors would like to thank prof.\ Bernard De Baets -- leader of the KERMIT Research Unit at Ghent University -- for years of friendly, fruitful cooperation. GSCA would not exist without his support and involvement.

\bibliography{papers-with-abstracts}

\end{document}
